\documentclass[times, 8pt,twocolumn]{article}
%\usepackage{ucs}
\usepackage{amsmath}
\usepackage{amsfonts}
%\usepackage{amssymb}
%\usepackage{glossary}
%\usepackage{ucs}
\usepackage{listings}
%\usepackage{fontenc}
\usepackage{graphicx}
\usepackage{float}
\usepackage{color}
%\usepackage{tipa}
\usepackage{url}
\usepackage{hyperref}
%\usepackage{wrapfig}
%\usepackage{subfig}
\usepackage[left=1cm, top=1cm, right=1cm, bottom=1cm, nohead, nofoot]{geometry}
\title {RMI-based chat application: Papinho}
\author{Andon Tchechmedjiev, Jander Nascimento}
 \linespread{0.9}
% \usepackage[small,compact]{titlesec}
 \addtolength{\parskip}{-1.59mm}
\lstset{
basicstyle=\footnotesize,
tabsize=1,
breaklines=true,
language=java
}
 
\begin{document}
{\Large {\bf Using JMS to gather resource usage from remote PCs}} \\
{\large {\bf {\it Andon Tchechmedjiev, Jander Nascimiento}}}
\section{Objective}
This prototype was developed intending to use Java Technology to obtain resource information locally and then dispatch these thought a JMS channel.
The information sent should be summarized to be according to the informations of all lower level nodes.
This communication structure is based on a tree format.
Overall, this prototype allows for the user to customize the tree by choosing the arity or the depth of the tree.

\section{Used API's}

Java Specification Request, or JSR, specify a completely detailed specification allowing that some software that follow the specification be able to run in any container that respects the JSR.
JSR 914 states the Java Messaging Service API specification, we have couple containers that implement this JSR, the we have chosen the OpenJMS.
Java Management Extension provides a dynamic and modular platform to monitor devices or applications connected to a network.
As JMX has a native support in Java, we just used the built-in support to get local information from the nodes. 

\section{Architecture}
\subsection{Applied design patterns}
\end{document}
